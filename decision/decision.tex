% this file is part of the Data Analysis Recipes project
% Copyright 2012 David W. Hogg

\documentclass[12pt,twoside,pdftex]{article}
\usepackage{amssymb}
\usepackage{amsmath}
\usepackage{mathrsfs}
\IfFileExists{./hogg_endnotes.sty}{\usepackage{./hogg_endnotes}}{}
\usepackage{natbib}
\usepackage{float}
\usepackage{graphicx}

%%Figure caption
\makeatletter
\newsavebox{\tempbox}
\newcommand{\@makefigcaption}[2]{%
\vspace{10pt}{#1.--- #2\par}}%
\renewcommand{\figure}{\let\@makecaption\@makefigcaption\@float{figure}}
\makeatother

\newcommand{\exampleplot}[1]{%
\begin{center}%
\includegraphics[width=0.5\textwidth]{#1}%
\end{center}%
}
\newcommand{\exampleplottwo}[2]{%
\begin{center}%
\includegraphics[width=0.5\textwidth]{#1}%
\includegraphics[width=0.5\textwidth]{#2}%
\end{center}%
}

\setlength{\emergencystretch}{2em}%No overflow

\newcommand{\notenglish}[1]{\textsl{#1}}
\newcommand{\aposteriori}{\notenglish{a~posteriori}}
\newcommand{\apriori}{\notenglish{a~priori}}
\newcommand{\adhoc}{\notenglish{ad~hoc}}
\newcommand{\etal}{\notenglish{et al.}}
\newcommand{\eg}{\notenglish{e.g.}}

\newcommand{\documentname}{document}
\newcommand{\documentnames}{\documentname s}
\newcommand{\sectionname}{Section}
\newcommand{\equationname}{equation}
\newcommand{\Equationname}{Equation}
\newcommand{\equationnames}{\equationname s}
\newcommand{\figurenames}{\figurename s}
\newcommand{\problemname}{Exercise}
\newcommand{\problemnames}{\problemname s}
\newcommand{\solutionname}{Solution}
\newcommand{\notename}{note}
\renewcommand{\and}{{\footnotesize{and}}}

\newcommand{\note}[1]{\endnote{#1}}
\def\enotesize{\normalsize}
\renewcommand{\thefootnote}{\fnsymbol{footnote}} % the ONE footnote needs this

\newcounter{problem}
\newenvironment{problem}{\paragraph{\problemname~\theproblem:}\refstepcounter{problem}}{}
\newcommand{\affil}[1]{{\footnotesize\textsl{#1}}}

% matrix stuff
\newcommand{\mmatrix}[1]{\boldsymbol{#1}}
\newcommand{\inverse}[1]{{#1}^{-1}}
\newcommand{\transpose}[1]{{#1}^{\scriptscriptstyle \top}}
\newcommand{\mA}{\mmatrix{A}}
\newcommand{\mAT}{\transpose{\mA}}
\newcommand{\mC}{\mmatrix{C}}
\newcommand{\mCinv}{\inverse{\mC}}
\newcommand{\mQ}{\mmatrix{Q}}
\newcommand{\mS}{\mmatrix{S}}
\newcommand{\mX}{\mmatrix{X}}
\newcommand{\mY}{\mmatrix{Y}}
\newcommand{\mYT}{\transpose{\mY}}
\newcommand{\mZ}{\mmatrix{Z}}
\newcommand{\vhat}{\mmatrix{\hat{v}}}

% parameter vectors
\newcommand{\parametervector}[1]{\mmatrix{#1}}
\newcommand{\pvtheta}{\parametervector{\theta}}

% set stuff
\newcommand{\setofall}[3]{\{{#1}\}_{{#2}}^{{#3}}}
\newcommand{\allq}{\setofall{q_i}{i=1}{N}}
\newcommand{\allx}{\setofall{x_i}{i=1}{N}}
\newcommand{\ally}{\setofall{y_i}{i=1}{N}}
\newcommand{\allxy}{\setofall{x_i,y_i}{i=1}{N}}
\newcommand{\allsigmay}{\setofall{\sigma_{yi}^2}{i=1}{N}}
\newcommand{\allS}{\setofall{\mS_i}{i=1}{N}}

% other random multiply used math symbols
\renewcommand{\d}{\mathrm{d}}
\newcommand{\dd}{\d}
\newcommand{\mean}[1]{\left<{#1}\right>}
\newcommand{\like}{\mathscr{L}}
\newcommand{\given}{\,|\,}


% header stuff
\renewcommand{\MakeUppercase}[1]{#1}
\pagestyle{myheadings}
\renewcommand{\sectionmark}[1]{\markright{\thesection.~#1}}
\markboth{Extracting information from images}{Introduction}

\begin{document}
\thispagestyle{plain}\raggedbottom
\section*{Data analysis recipes:\\
  Making decisions}

\footnotetext{%
  The \notename s begin on page~\pageref{note:first}, including the
  license\note{\label{note:first}%
    Copyright 2012 David W. Hogg (NYU).  Right now this document
    (though publicly available) is NOT FOR DISTRIBUTION.  Eventually,
    and once it is ready, I will release it with the license ``You may
    copy and distribute this document provided that you make no
    changes to it whatsoever''.}
  and the acknowledgements\note{%
    It is a pleasure to thank
      Jo Bovy (IAS),
      Dan Foreman-Mackey (NYU),
      Dustin Lang (CMU), and
      Sam Roweis (deceased)
    for discussions and comments.  This
    research was partially supported by the US National Aeronautics
    and Space Administration and National Science Foundation.}}

\noindent
David~W.~Hogg\\
\affil{Center~for~Cosmology~and~Particle~Physics, Department~of~Physics, New York University}\\
and~\affil{Max-Planck-Institut f\"ur Astronomie, Heidelberg}
%% \\[1ex]
%% A. Nother Author
%% \affil{Oxford University}

\begin{abstract}
Probabilistic inference at its best gives probabilistic information
about parameters and models.  In many cases of importance---as with
experimental design or purchasing or delivery of results---it is
necessary to make a hard decision or choice.  It is almost never the
right move to choose the most probable model or option; it is rare
that the most probable is the option that delivers the most benefit to
the decider.  Instead the decider ought to specify a utility function
and work to optimize expected utility, or minimize expected loss, or
minimize maximal loss.  In real-world situations of building
experiments or operating an organization, to be generally useful the
utility must be specified in (the equivalent of) currency (dollar)
units.  The utility should be an approximation to the decider's
\emph{long-term future discounted free cash flow}.
\end{abstract}

DETF introduction

\section{What is a decision?}

\section{Utility}

minimax and utility maximization

\section{Long-term future discounted free cash flow}

Long-term future.

Discounted.

Free cash flow.

Assumptions under which this is possible.

\clearpage
\markright{Notes}\theendnotes

\clearpage
\begin{thebibliography}{}\markright{References}\raggedright
\bibitem[Hogg \etal(2010a)]{straightline}
  Hogg,~D.~W., Bovy,~J., \& Lang,~D., 2010a,
  ``Data analysis recipes:\ Fitting a model to data'', arXiv:1008.4686
\end{thebibliography}

\end{document}
