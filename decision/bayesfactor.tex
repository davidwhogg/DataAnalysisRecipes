\documentclass[12pt]{article}

\begin{document}

\section*{Don't ever calculate the fully marginalized likelihood!}

\noindent
David W. Hogg (NYU) and friends

\paragraph{abstract:}
In probabilistic inference, the relative probabilities of two mutually
exclusive parameterized models can be obtained by an operation that
involves marginalizing the likelihood for each model over the entire
parameter space, using the prior as a measure for the integration.
This marginalization produces the fully marginalized likelihood,
sometimes called the ``Bayes Factor'' or ``evidence''.  Here we argue
that performing this integral in is almost never a good idea.  The
argument involves the ideas that (a)~the integral is extremely
challenging to perform numerically in many instances, (b)~the integral
is extremely prior-dependent, and in many cases priors are not
understood at a precision that justifies the calculation, (c)~the
integral is usually performed in order to aid in decision-making
(model selection), where a utility ought to be integrated along with
the likelihood, (d) utilities are also rarely known precisely, and
(e)~there are approximate or heuristic methods for decision making
that are just as (im)precise but require no expensive integration.
The one place where calculation of the marginalized likelihood is
necessary is in creating posterior mixtures of distinct models; this
is rarely the context in which the integral is being calculated but
even there, when priors are not precisely understood, the mixture
weights will not be correct, despite the spirit-crushing calculation.

\section{Why marginalize the likelihood?}

% Set up a straw-man example where it seems like a very, very good idea.
% Example should lead to decision-making.
% Example should have many parameters.
% transiting exoplanet vs blended eclipsing binary?

\section{Computational and algorithmic cost}

\section{Prior mis-specification}

Hierarchical inference and pooling has to be harmed for priors to be well specified.

\section{Utility matters}

The utility is not (just) a model-level object.
It is also a parameter-level object.
Thus, for \emph{decision-making},
  the integrand of the marginalization has to change to include the parameter-dependent utility.

\section{Utility mis-specification}

\section{Heurisic methods}

\section{Sufficient conditions}


\paragraph{Acknowledgements:}
Jim Berger,
Andrew Gelman,
Sam Roweis

\section*{Bibliography}

\end{document}
