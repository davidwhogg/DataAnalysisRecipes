% This file is part of the Data Analysis Recipes project.
% Copyright 2020 the author.

% to-do
% -----
% - write outline?
% - possibly switch formatting to the Gaussian product format?

\documentclass[12pt,twoside,pdftex]{article}
\usepackage{amssymb,amsmath,mathrsfs,../deluxetable,../hogg_endnotes,natbib}
\usepackage{url}
\usepackage{amssymb}
\usepackage{amsmath}
\usepackage{mathrsfs}
\IfFileExists{./hogg_endnotes.sty}{\usepackage{./hogg_endnotes}}{}
\usepackage{natbib}
\usepackage{float}
\usepackage{graphicx}

%%Figure caption
\makeatletter
\newsavebox{\tempbox}
\newcommand{\@makefigcaption}[2]{%
\vspace{10pt}{#1.--- #2\par}}%
\renewcommand{\figure}{\let\@makecaption\@makefigcaption\@float{figure}}
\makeatother

\newcommand{\exampleplot}[1]{%
\begin{center}%
\includegraphics[width=0.5\textwidth]{#1}%
\end{center}%
}
\newcommand{\exampleplottwo}[2]{%
\begin{center}%
\includegraphics[width=0.5\textwidth]{#1}%
\includegraphics[width=0.5\textwidth]{#2}%
\end{center}%
}

\setlength{\emergencystretch}{2em}%No overflow

\newcommand{\notenglish}[1]{\textsl{#1}}
\newcommand{\aposteriori}{\notenglish{a~posteriori}}
\newcommand{\apriori}{\notenglish{a~priori}}
\newcommand{\adhoc}{\notenglish{ad~hoc}}
\newcommand{\etal}{\notenglish{et al.}}
\newcommand{\eg}{\notenglish{e.g.}}

\newcommand{\documentname}{document}
\newcommand{\documentnames}{\documentname s}
\newcommand{\sectionname}{Section}
\newcommand{\equationname}{equation}
\newcommand{\Equationname}{Equation}
\newcommand{\equationnames}{\equationname s}
\newcommand{\figurenames}{\figurename s}
\newcommand{\problemname}{Exercise}
\newcommand{\problemnames}{\problemname s}
\newcommand{\solutionname}{Solution}
\newcommand{\notename}{note}
\renewcommand{\and}{{\footnotesize{and}}}

\newcommand{\note}[1]{\endnote{#1}}
\def\enotesize{\normalsize}
\renewcommand{\thefootnote}{\fnsymbol{footnote}} % the ONE footnote needs this

\newcounter{problem}
\newenvironment{problem}{\paragraph{\problemname~\theproblem:}\refstepcounter{problem}}{}
\newcommand{\affil}[1]{{\footnotesize\textsl{#1}}}

% matrix stuff
\newcommand{\mmatrix}[1]{\boldsymbol{#1}}
\newcommand{\inverse}[1]{{#1}^{-1}}
\newcommand{\transpose}[1]{{#1}^{\scriptscriptstyle \top}}
\newcommand{\mA}{\mmatrix{A}}
\newcommand{\mAT}{\transpose{\mA}}
\newcommand{\mC}{\mmatrix{C}}
\newcommand{\mCinv}{\inverse{\mC}}
\newcommand{\mQ}{\mmatrix{Q}}
\newcommand{\mS}{\mmatrix{S}}
\newcommand{\mX}{\mmatrix{X}}
\newcommand{\mY}{\mmatrix{Y}}
\newcommand{\mYT}{\transpose{\mY}}
\newcommand{\mZ}{\mmatrix{Z}}
\newcommand{\vhat}{\mmatrix{\hat{v}}}

% parameter vectors
\newcommand{\parametervector}[1]{\mmatrix{#1}}
\newcommand{\pvtheta}{\parametervector{\theta}}

% set stuff
\newcommand{\setofall}[3]{\{{#1}\}_{{#2}}^{{#3}}}
\newcommand{\allq}{\setofall{q_i}{i=1}{N}}
\newcommand{\allx}{\setofall{x_i}{i=1}{N}}
\newcommand{\ally}{\setofall{y_i}{i=1}{N}}
\newcommand{\allxy}{\setofall{x_i,y_i}{i=1}{N}}
\newcommand{\allsigmay}{\setofall{\sigma_{yi}^2}{i=1}{N}}
\newcommand{\allS}{\setofall{\mS_i}{i=1}{N}}

% other random multiply used math symbols
\renewcommand{\d}{\mathrm{d}}
\newcommand{\dd}{\d}
\newcommand{\mean}[1]{\left<{#1}\right>}
\newcommand{\like}{\mathscr{L}}
\newcommand{\given}{\,|\,}

\usepackage{float,graphicx}

% header stuff
\renewcommand{\MakeUppercase}[1]{#1}
\pagestyle{myheadings}
\renewcommand{\sectionmark}[1]{\markright{\thesection.~#1}}
\markboth{Foo and Bar}{}

\begin{document}
\thispagestyle{plain}\raggedbottom
\section*{\raggedright
  Everything I know about data analysis, all in one place\footnotemark}

\footnotetext{%
  The \notename s begin on page~\pageref{note:first}, including the
  license\note{\label{note:first}%
    Copyright 2020 by the author.  You may copy and distribute this
    document provided that you make no changes to it whatsoever.}  and
  the acknowledgements\note{%
    Above all I owe thanks to all my students, postdocs, and collaborators,
    from whom I learned all this---by doing it.
    This research was supported by NASA, the NSF, the Simons Foundation,
    the Moore Foundation, the Sloan Foundation, and the 
    Alexander von Humboldt Foundation.}}

\noindent
David~W.~Hogg\\
\affil{Center~for~Cosmology~and~Particle~Physics, Department~of~Physics, New York University}\\
\affil{Max-Planck-Institut f\"ur Astronomie, Heidelberg}\\
\affil{Flatiron Institute, a division of the Simons Foundation}

\begin{abstract}
  Here are the points I will make: Foo and bar.
\end{abstract}

I learned all this by doing it, in collaboration with others!

\paragraph{Science is a literature.}
Introduce ideas here of humanity, human choices, subjectivity, decision
theory, LTFDFCF, pragmatism, your audience, your career, and so on.

\paragraph{Be pragmatic.}
foo

\paragraph{Think about how the data were generated.}
Connect to causal structure, and hierarchical models.

\paragraph{State very clearly your assumptions.}
Once your assumptions are well specified, your method will flow naturally.
Note that you can't state \emph{all} of your assumptions.

\paragraph{Bring the model to the data.}
Don't correct the data, correct the model! Unless it gets very expensive.
Connects to pragmatism.

\paragraph{Binning is sinning.}
Any time you are binning, there is a regression-like alternative.
This also connects to bringing the model to the data!

\paragraph{Don't select models when you can estimate parameters.}
Related to the general point that continuous problems are harder
than discrete problems.

\paragraph{Compare to information theory when you can.}

\paragraph{Use empirical uncertainty estimates and model validation.}
Bring up the mild conflict between this and the information-theory
point.

\paragraph{Don't make stuff up.}
Either use known methods, or else tailor your method to your carefully
specified beliefs (assumptions).

\paragraph{Stay close to the methodologists.}

\paragraph{Principled model selection requires a theory of decision-making.}

\paragraph{Don't commit to a philosophy.}
Use Bayes and frequentism as it makes sense.

\paragraph{Data analysis is subjective.}
Use this to bring together various threads from the above.

\clearpage
\markright{Notes}\theendnotes

\clearpage
\begin{thebibliography}{}\markright{References}
\bibitem[Bovy, Hogg, \& Roweis(2009)]{bovy}
  Bovy,~J., Hogg,~D.~W., \& Roweis, S.~T., 2009,
  Extreme deconvolution: inferring complete distribution functions from noisy, heterogeneous, and incomplete observations, 
  arXiv:0905.2979 [stat.ME]
\bibitem[Jaynes(2003)]{jaynes}
  Jaynes,~E.~T., 2003,
  \textit{Probability theory: the logic of science} (Cambridge University Press)
\bibitem[Gilks, Richardson, \& Spiegelhalter(1995)]{gilksmcmc}
  Gilks,~W.~R., Richardson,~S., \& Spiegelhalter,~D., 1995,
  \textit{Markov chain Monte Carlo in practice: interdisciplinary statistics} (Chapman \& Hall/CRC)
\bibitem[Hampel \etal(1986)]{hampel}
  Hampel,~F.~R., Ronchetti,~E.~M., Rousseeuw,~P.~J., \& Stahel,~W.~A., 1986, 
  \textit{Robust statistics: the approach based on influence functions} (New York: Wiley)
\bibitem[Isobe \etal(1990)]{isobe90}
  Isobe,~T., Feigelson, E.~D., Akritas,~M.~G., \& Babu,~G.~J., 1990,
  Linear regression in astronomy,
  \textit{Astrophysical Journal} \textbf{364} 104
\bibitem[Kelly(2007)]{kelly07}
  Kelly,~B.~C., 2007,
  Some aspects of measurement error in linear regression of astronomical data,
  \textit{Astrophysical Journal} \textbf{665} 1489
\bibitem[Mackay(2003)]{mackay}
  Mackay,~D.~J.~C., 2003,
  \textit{Information theory, inference, and learning algorithms} (Cambridge University Press)
\bibitem[Neal(2003)]{neal2003a}
  Neal.,~R.~M., 2003,
  Slice sampling,
  \textit{Annals of Statistics}, \textbf{31}(3), 705
\bibitem[Press(1997)]{pressH0}
  Press,~W.~H., 1997,
  Understanding data better with Bayesian and global statistical methods,
  in \textit{Unsolved problems in astrophysics,}
  eds. Bahcall,~J.~N. \& Ostriker,~J.~P. (Princeton University Press)
  49--60
\bibitem[Press \etal(2007)]{press}
  Press,~W.~H., Teukolsky,~S.~A., Vetterling,~W.~T., \& Flannery,~B.~P., 2007,
  \textit{Numerical recipes: the art of scientific computing} (Cambridge University Press)
\bibitem[Sivia \& Skilling(2006)]{sivia}
  Sivia,~D.~S. \& Skilling,~J., 2006,
  \textit{Data analysis: a Bayesian tutorial} (Oxford University Press)
\end{thebibliography}

\end{document}
