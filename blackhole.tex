\documentclass[12pt,preprint]{aastex}
\newcounter{address}
\setcounter{address}{1}
\newcommand{\mbulge}{m_{\mathrm{b}}}
\newcommand{\mbh}{m_{\mathrm{BH}}}
\newcommand{\slope}{n}
\begin{document}
\title{The functional form of the black-hole--bulge-mass relation}
\author{David~W.~Hogg\altaffilmark{\ref{CCPP},\ref{MPIA},\ref{email}},
        Dustin~Lang\altaffilmark{\ref{UofT},\ref{Princeton}},
        Jo~Bovy\altaffilmark{\ref{CCPP}}}
\altaffiltext{\theaddress}{\label{CCPP}\refstepcounter{address}
  Center for Cosmology and Particle Physics, Department of Physics, New York University, 4 Washington Place, New York, NY 10003, USA}
\altaffiltext{\theaddress}{\label{MPIA}\refstepcounter{address}
  Max-Planck-Institut f\"ur Astronomie, K\"onigstuhl 17, D-69117 Heidelberg, Germany}
\altaffiltext{\theaddress}{\label{email}\refstepcounter{address}
  Correspondence should be addressed to david.hogg@nyu.edu~.}
\altaffiltext{\theaddress}{\stepcounter{address}\label{UofT}
  Department of Computer Science, University of Toronto, 6 King's College Road, Toronto, Ontario, M5S~3G4 Canada}
\altaffiltext{\theaddress}{\label{Princeton}\refstepcounter{address}
  Princeton University Observatory, Princeton NJ 08544}

\begin{abstract}
We use justifiable likelihood and bayesian methods to measure the
power-law relationship $\mbh\propto\mbulge^\slope$ between bulge mass
$\mbulge$ and black-hole mass $\mbh$ in nearby galaxies.  We consider
cases in which the slope of the relation is fixed at unity (bulge mass
linearly proportional to black-hole mass) and cases in which it is
free; we consider a model in which the relationship has finite
gaussian intrinsic scatter and a model in which the relationship has
no scatter but some galaxies are permitted to be ``outliers''.  We
find that at unit slope, the ratio of bulge mass to black-hole mass is
between $X$ and $Y$ (95-percent confidence).  We find that when the
slope $\slope$ is permitted to vary, it is between $U$ and $V$
(95-percent confidence).  In the intrinsic-scatter model,
marginalizing over all other parameters, we find that the intrinsic
scatter in the power-law relation is between $A$ and $B$~dex
(95-percent confidence), where the scatter is measured perpendicular
to the relation in log--log space.  In the outlier model,
marginalizing over all other parameters, we find that the
highest-probability outliers from the relationship are P and Q.
\end{abstract}

The whatever.

\acknowledgments It is a pleasure to thank Hans-Walter Rix for. JB and
DWH were partially supported by NASA (grant NNX08AJ48G).  JB was
partially supported by New York University's Horizon fellowship.  DWH
is a research fellow of the Alexander von Humboldt Foundation of
Germany.

\end{document}
