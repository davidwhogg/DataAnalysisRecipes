\documentclass[12pt]{article}
\begin{document}
\section*{Data analysis recipes:\ \\
  Fitting a straight line to data\footnote{
    Copyright 2009 David~W.~Hogg (david.hogg@nyu.edu).  You may copy
    and distribute this document provided that you make no changes to
    it whatsoever.}}

\noindent
David~W.~Hogg\\
\textsl{Center~for~Cosmology~and~Particle~Physics, Department~of~Physics,\\
New~York~University}\\
\texttt{david.hogg@nyu.edu}

\begin{abstract}
  In excruciating detail, I go through all of the considerations
  involved in fitting a straight line to a set of points in a
  two-dimensional plane.  Standard chi-squared fitting is only
  appropriate when there is a dimension along which the data points
  have negligible uncertainties; this condition is rarely met in
  practice.  In addition to considering cases of general,
  heterogeneous, and arbitrarily covariant two-dimensional
  uncertainties, I also look at situations in which there are large
  outliers, upper or lower limits on some points, unknown
  uncertainties, and unknown but expected intrinsic scatter in the
  linear relationship being fit.  Above all I emphasize the importance
  of choosing a justified scalar objective, and recommend separating
  that decision from any decisions about the details of optimization.
\end{abstract}

Common problem.

No consensus on generalizations.

Absurdities in the literature.

Robot science: I want methods that are so reliable that they can be
performed automatically by unsupervised robots on millions of
problems.

\section{Standard practice}

Note that even when the conditions of standard practice are met, it is
\emph{still} often done wrong!

\section{Arbitrary two-dimensional uncertainties}

\section{Comments on non-gaussian errors}

\section{Robustness to outliers}

Criticize sigma-clipping relative to having a known scalar objective.
Sigma-clipping is a procedure the end point of which depends on the
starting point.

\section{Limits}

\section{Unknown datapoint uncertainties}

\section{Intrinsic scatter}

\section{Discussion}

\end{document}
