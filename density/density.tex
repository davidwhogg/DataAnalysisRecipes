% This file is part of the Data Analysis Recipes project.
% Copyright 2013 the author(s).

% to-do
% -----
% - outline
% - write

\documentclass[12pt,twoside]{article}
\usepackage{amssymb}
\usepackage{amsmath}
\usepackage{mathrsfs}
\IfFileExists{./hogg_endnotes.sty}{\usepackage{./hogg_endnotes}}{}
\usepackage{natbib}
\usepackage{float}
\usepackage{graphicx}

%%Figure caption
\makeatletter
\newsavebox{\tempbox}
\newcommand{\@makefigcaption}[2]{%
\vspace{10pt}{#1.--- #2\par}}%
\renewcommand{\figure}{\let\@makecaption\@makefigcaption\@float{figure}}
\makeatother

\newcommand{\exampleplot}[1]{%
\begin{center}%
\includegraphics[width=0.5\textwidth]{#1}%
\end{center}%
}
\newcommand{\exampleplottwo}[2]{%
\begin{center}%
\includegraphics[width=0.5\textwidth]{#1}%
\includegraphics[width=0.5\textwidth]{#2}%
\end{center}%
}

\setlength{\emergencystretch}{2em}%No overflow

\newcommand{\notenglish}[1]{\textsl{#1}}
\newcommand{\aposteriori}{\notenglish{a~posteriori}}
\newcommand{\apriori}{\notenglish{a~priori}}
\newcommand{\adhoc}{\notenglish{ad~hoc}}
\newcommand{\etal}{\notenglish{et al.}}
\newcommand{\eg}{\notenglish{e.g.}}

\newcommand{\documentname}{document}
\newcommand{\documentnames}{\documentname s}
\newcommand{\sectionname}{Section}
\newcommand{\equationname}{equation}
\newcommand{\Equationname}{Equation}
\newcommand{\equationnames}{\equationname s}
\newcommand{\figurenames}{\figurename s}
\newcommand{\problemname}{Exercise}
\newcommand{\problemnames}{\problemname s}
\newcommand{\solutionname}{Solution}
\newcommand{\notename}{note}
\renewcommand{\and}{{\footnotesize{and}}}

\newcommand{\note}[1]{\endnote{#1}}
\def\enotesize{\normalsize}
\renewcommand{\thefootnote}{\fnsymbol{footnote}} % the ONE footnote needs this

\newcounter{problem}
\newenvironment{problem}{\paragraph{\problemname~\theproblem:}\refstepcounter{problem}}{}
\newcommand{\affil}[1]{{\footnotesize\textsl{#1}}}

% matrix stuff
\newcommand{\mmatrix}[1]{\boldsymbol{#1}}
\newcommand{\inverse}[1]{{#1}^{-1}}
\newcommand{\transpose}[1]{{#1}^{\scriptscriptstyle \top}}
\newcommand{\mA}{\mmatrix{A}}
\newcommand{\mAT}{\transpose{\mA}}
\newcommand{\mC}{\mmatrix{C}}
\newcommand{\mCinv}{\inverse{\mC}}
\newcommand{\mQ}{\mmatrix{Q}}
\newcommand{\mS}{\mmatrix{S}}
\newcommand{\mX}{\mmatrix{X}}
\newcommand{\mY}{\mmatrix{Y}}
\newcommand{\mYT}{\transpose{\mY}}
\newcommand{\mZ}{\mmatrix{Z}}
\newcommand{\vhat}{\mmatrix{\hat{v}}}

% parameter vectors
\newcommand{\parametervector}[1]{\mmatrix{#1}}
\newcommand{\pvtheta}{\parametervector{\theta}}

% set stuff
\newcommand{\setofall}[3]{\{{#1}\}_{{#2}}^{{#3}}}
\newcommand{\allq}{\setofall{q_i}{i=1}{N}}
\newcommand{\allx}{\setofall{x_i}{i=1}{N}}
\newcommand{\ally}{\setofall{y_i}{i=1}{N}}
\newcommand{\allxy}{\setofall{x_i,y_i}{i=1}{N}}
\newcommand{\allsigmay}{\setofall{\sigma_{yi}^2}{i=1}{N}}
\newcommand{\allS}{\setofall{\mS_i}{i=1}{N}}

% other random multiply used math symbols
\renewcommand{\d}{\mathrm{d}}
\newcommand{\dd}{\d}
\newcommand{\mean}[1]{\left<{#1}\right>}
\newcommand{\like}{\mathscr{L}}
\newcommand{\given}{\,|\,}


% header stuff
\renewcommand{\MakeUppercase}[1]{#1}
\pagestyle{myheadings}
\renewcommand{\sectionmark}[1]{\markright{\thesection.~#1}}
\markboth{Density estimation from noisy samples}{}

\newcommand{\data}{D}
\newcommand{\pars}{\theta}
\newcommand{\cv}{\mathrm{CV}}
\newcommand{\pcv}{p_{\cv}}

\begin{document}
\thispagestyle{plain}\raggedbottom
\section*{Data analysis recipes:\ \\
  Density estimation from noisy samples\footnotemark}

\footnotetext{The \notename s begin on page~\pageref{note:first},
  including the license\note{\label{note:first} Copyright 2013 by the
    author.  You may copy and distribute this document provided that
    you make no changes to it whatsoever.}  and the
  acknowledgements\note{It is a pleasure to thank
      Jo Bovy (IAS),
      Dustin Lang (CMU),
      Dan Foreman-Mackey (NYU),
      Hans-Walter Rix (MPIA),
      Sam Roweis (deceased), and
      Bernhard Sch\"olkopf (MPI-IS)
    for valuable comments and discussions.  This research was
    partially supported by NASA (ADP grant NNX08AJ48G), NSF (grant
    AST-0908357), and a Research Fellowship of the Alexander von
    Humboldt Foundation.  This research made use of the Python
    programming language and the open-source Python packages scipy,
    numpy, and matplotlib.}.}

\noindent
David~W.~Hogg\\
\affil{Center~for~Cosmology~and~Particle~Physics, Department~of~Physics, New~York~University}\\
\affil{Max-Planck-Institut f\"ur Astronomie, Heidelberg}
%% \\[1ex]
%% Jo~Bovy\\
%% \affil{Institute for Advanced Study, Princeton}
%% \\[1ex]
%% Dustin~Lang\\
%% \affil{Princeton University Observatory}\\
%% \affil{Department of Physics, Carnegie Mellon University}

\begin{abstract}
Frequently we have some repeated measurement of a quantity $Q$, one
measurement of $Q$ per object in some sample of objects; we often want
to know the distribution of this quantity in the population from which
the objects are drawn.  The simplest idea is to make a histogram of
the measurements, or estimates, of the quantity.  The problem with
this histogram approach is that, because the measurements are
necessarily noisy, the empirical histogram of estimates is always
broader or less informative than the true distribution for $Q$.  If
each measurement of $Q$ is based on a likelihood function that
includes an accurate noise model---even in the case that the noise is
different for every measurement---it is possible to replace the simple
histogramming with a hierarchical inference that obtains a
marginalized likelihood (or posterior probability) for the true
distribution of $Q$ that \emph{would have been observed} in a sample
with much less noise per measurement.  The hierarchical inference is a
kind of deconvolution, because it infers a more informative
distribution from less-informative samples, but it proceeds by
forward-modeling the data generation process.  We provide some
examples and more strong advice: Investigators sometimes create even
\emph{less} informative (broader) proxies for the distribution than
the empirical histogram of estimates; all such methods are bad.
\end{abstract}

\clearpage
\markright{Notes}\theendnotes

\clearpage
\begin{thebibliography}{}\markright{References}
\bibitem[Bovy, Hogg, \& Roweis(2009)]{bovy}
  Bovy,~J., Hogg,~D.~W., \& Roweis, S.~T., 2009,
  Extreme deconvolution: inferring complete distribution functions from noisy, heterogeneous, and incomplete observations, 
  arXiv:0905.2979 [stat.ME]
\end{thebibliography}

\end{document}
